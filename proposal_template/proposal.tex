%----------------------------------------------------------------------------------------
%	PACKAGES AND OTHER DOCUMENT CONFIGURATIONS
%----------------------------------------------------------------------------------------

\documentclass[11pt]{article}
%\input{mydef.tex}
\usepackage{fancyhdr} % Required for custom headers
\usepackage{lastpage} % Required to determine the last page for the footer
\usepackage{extramarks} % Required for headers and footers
\usepackage[usenames,dvipsnames]{color} % Required for custom colors
\usepackage{graphicx} % Required to insert images
\usepackage{listings} % Required for insertion of code
\usepackage{courier} % Required for the courier font
\usepackage{amssymb,amsmath}
\usepackage{amsfonts}
\usepackage{mathtools}
\usepackage{subfigure}
\usepackage{enumitem}
\usepackage{bm}
\usepackage{url}
\usepackage[stable]{footmisc}
\usepackage{booktabs}
\usepackage[square]{natbib}
\usepackage{indentfirst}
\usepackage[colorlinks, linkcolor=red, anchorcolor=purple, citecolor=blue]{hyperref}
\usepackage{hyperref}

\usepackage{multicol}
\setlength{\columnsep}{1cm}

% Margins
\topmargin=-0.45in
\evensidemargin=0in
\oddsidemargin=0in
\textwidth=6.5in
\textheight=9.0in
\headsep=0.25in
\setlength{\headheight}{13.6pt}
\linespread{1.1} % Line spacing

% Set up the header and footer
\pagestyle{fancy}
\lhead{CS 57800: Statistical Machine Learning} % Top left header
\chead{}
\rhead{Project Proposal} % Top right header
\lfoot{} % Bottom left footer
\cfoot{} % Bottom center footer
\rfoot{Page\ \thepage\ of\ \protect\pageref{LastPage}} % Bottom right footer
\renewcommand\headrulewidth{0.4pt} % Size of the header rule
\renewcommand\footrulewidth{0.4pt} % Size of the footer rule

\renewcommand*{\thefootnote}{\fnsymbol{footnote}}

\setlength{\parskip}{.2\baselineskip}
%\setlength\parindent{0pt} % Removes all indentation from paragraphs

\title{
\textbf{Project Proposal} \\ \textsc{\textit{Sensing Semantic Information from Mobile Social Networks}} \\
}

\author{
	\textbf{\textit{Ting Zhang and Wen Yi}} \\
	School of Industiral Engineering, School of Electrical and Computer Engineering\\
	\texttt{zhan1013@purdue.edu, yi35@purdue.edu}
}

\date{\today}

%----------------------------------------------------------------------------------------

\begin{document}

\maketitle

%\thispagestyle{empty}

\section{Introduction}
Human society consists of extensive communications and interactions between individuals, via the use of mobile sensors, such as mobile phones, tablets and GPS. The understanding of individual relations from these sensors, can greatly facilitate and promote the interactions between individuals. 
For example, listing phone contacts in semantic orders according to the time and location when a person wants to make a phone call, would save the person both time and memory load, to find a specific contact from a phone book with tons of contacts based on alphabetic order. 
Further more, by understanding the friendship network from the global view, we can group people in the whole network into small communities in which people have closer friendships. This may serve as a friend grouping suggestion function inside the cellphone contact managing software.
Therefore, in this project, we will focus on the inference of friend relationships with the detection of communities using data collected from mobile phones. Detailed explanation of dataset is stated in section \ref{Data and Evaluation plan}.

\section{Related work}
\subsection{Relationship Inference}
Relationship inference in social networks has been studied in various fields and domains. In this context, we refer to friendship inference between pairs of individuals. Representing social networks with topology structures provides insights to predict relationships  between individuals based on topology and probability distribution of the links in the topology. \cite{liben-nowell_link-prediction_2007} proposed different measurements to compute the ''similarity'' between two nodes (individuals) in the graph, including the distance between two nodes, number of shared neighbors, and ''meta-approaches'' that integrate different measurements. Beyond topological structures, individual attribute and context information have also been utilized to facilitate the construction of relations between individuals. In the study from \cite{taskar_link_2004}, correlations between individuals were constructed using user attributes with relational Markov Networks. For instance, they proposed a transitivity pattern that is useful in relationship prediction, where the presence of A-B relation and B-C relation promotes the probability of A-C relation. Context information, such as locations and periods of time, has also shown potentials to predict social ties. Using location information alone may not be a sufficient predictor. In \cite{crandall_inferring_2010}'s work, only 0.1\% of the relations were predicted with a confidence of 60\%. However, when network structure are analyzed together with location information, over 90\% friendship were detected with confidence over 80\%, illustrated in the study from \cite{sadilek_finding_2012}. Although location information alone is not a good indicator for friendship inference, a number of researches have indicated the importance between social ties and distance. The integration of location information and other features are also proved to be of high accuracy in friendship inference.

\subsection{Community Detection}
While friendship inferring are engaged to predict the  local relationships between individuals, community detection, from a global aspect of view, groups people into smaller subgroups with tighter relationships. 
In previous research such as the studies from \cite{mislove_you_2010} and \cite{xie_community_2011}, from people's friendship conditions in social network, community detection groups the people into overlap or distinct ''communities'', while the members of the same community known quite a few others from the same community or the members share the similar characteristics. Further more, studies from  \cite{mislove_you_2010} showing that, community detection can be used to infer the vacant profile information of people based on the profile information from other members in the same community. 

From the algorithm implementation view, the social network is usually defined as an undirected unweighted graph, while the vertexes represent the individuals, the edges represents the friendship between individuals. 
To change the social network into communities, various of problem statement and algorithm were posed. In the study of \cite{girvan_community_2002}, the problem statement was '' removing the edges which is likely to be the friendship across communities, until the left graph divided into unconnected components, while each component represent one community''. On the other hand, in the study of  \cite{newman_fast_2004}, the problem statement becomes '' start with each individual as a community, then merge the community into another one with greatest increase on the global modularity''.

Besides the classic community detection model, as our social network may contains friendship duplicate between individuals such as multiple phone calls between pair of individuals, we may also try to apply clustering algorithms on weighted graph representations.


\section{Problem formulation}
\textit{Describe your project as a machine learning problem, identify inputs objects, labels, possible features}
\subsection{Community Detection}
For community detection problem, the ideal community result we want to get is the group belongings of the individual. As provided by the dataset we use, there were 94 volunteer participate the entire data collecting process. 68 of them were colleagues working in the same building on campus (90\% graduate students, 10\% staff) while the remaining 26 participants were incoming students at the university’s business school. The above identification information for each individual is stored inside the dataset.
For the problem statement, we define the network between the participants as two distinct social network, phone call network and Bluetooth network. As both of the networks may contain duplicate friendships, we may firstly turns the relation condition between people as a binary variable, 

\section{Data and Evaluation Plan} \label{Data and Evaluation plan}
We will use a reality mining dataset from MIT media lab \cite{eagle_inferring_2009}. The dataset consists of phone logs of 94 subjects from September 2004 to June 2005. Among these 94 subjects, 68 were colleagues working in the same building (90\% were graduate students, while 10\% were staff). The remaining 26 subjects were incoming students from the business school. The dataset was collected from Nokia 6600 phones programmed to automatically run a log application as background process, including phone log, bluetooth and location. The format of each log is summarized as following:
\begin{itemize}
\item[*]
Phone log: (TIME)  20060720T211505  (DESCRIPTION)  Voice  call  (DIRECTION)  Outgoing  (DURATION 
seconds) 23 (NUMBER) 6175559821
\item[*]
Bluetooth: (TIME) 20060721T111222 devices: 000e6d2a3564 [Amy’s Phone]000e6d2b06ea [Jon’s PalmPilot] 
\item[*]
Location:(TIME) 20060721T111222 (CELL AREA) 24127, (CELL TOWER) 111, (SERVICE PROVIDER) AT\&T 
Wirel (USER DEFINED LOCATION NAME) My Office
\end{itemize}

\textit{How will you evaluate your algorithm? What is a reasonable baseline?}



\section*{Submission Instructions:} 
\textit{delete this section when submitting}

You are required to use \LaTeX \, to type your solutions to questions, and report of your programing as well. Other formats of submission will \textbf{not} be accepted. A template named ``homework.tex" is also provided for your convenience.\\

After logging into data.cs.purdue.edu (physically go to the lab or use ssh remotely, you are all granted the accounts to CS data machines during this class), please follow these steps to submit your assignment:
\begin{enumerate}
	\item Make a directory named \textit{`your Name\_your Surname'} and copy all of your files there.
	\item While in the upper level directory (if the files are in /homes/dan/dan\_goldwasser, go to/homes/dan), execute the following command:\\

	\texttt{turnin -c cs578 -p PROPOSAL *your\_folder\_name*}
		
	(e.g. your prof would use: \texttt{turnin -c cs578 -p PROPOSAL dan\_goldwasser} to submit his work)\\
		
	Keep in mind that old submissions are overwritten with new ones whenever you execute this
command.\\

	\item You can verify the contents of your submission by executing the following command:\\
	
	\texttt{turnin -v -c cs578 -p PROPOSAL\\}

	Do \textbf{not} forget the -v flag here, as otherwise your submission would be replaced with an empty
one.
\end{enumerate}

%\nocite{*}
\bibliographystyle{plainnat}
%\bibliographystyle{ieeetr}
\bibliography{CS578_project}

\end{document}

%----------------------------------------------------------------------------------------
%	PACKAGES AND OTHER DOCUMENT CONFIGURATIONS
%----------------------------------------------------------------------------------------

\documentclass[11pt]{article}
%\input{mydef.tex}
\usepackage{fancyhdr} % Required for custom headers
\usepackage{lastpage} % Required to determine the last page for the footer
\usepackage{extramarks} % Required for headers and footers
\usepackage[usenames,dvipsnames]{color} % Required for custom colors
\usepackage{graphicx} % Required to insert images
\usepackage{listings} % Required for insertion of code
\usepackage{courier} % Required for the courier font
\usepackage{amssymb,amsmath}
\usepackage{amsfonts}
\usepackage{mathtools}
\usepackage{subfigure}
\usepackage{enumitem}
\usepackage{bm}
\usepackage{url}
\usepackage[stable]{footmisc}
\usepackage{booktabs}
\usepackage[square]{natbib}
\usepackage{indentfirst}
\usepackage[colorlinks, linkcolor=red, anchorcolor=purple, citecolor=blue]{hyperref}
\usepackage{hyperref}

\usepackage{multicol}
\setlength{\columnsep}{1cm}

% Margins
\topmargin=-0.45in
\evensidemargin=0in
\oddsidemargin=0in
\textwidth=6.5in
\textheight=9.0in
\headsep=0.25in
\setlength{\headheight}{13.6pt}
\linespread{1.1} % Line spacing

% Set up the header and footer
\pagestyle{fancy}
\lhead{CS 57800: Statistical Machine Learning} % Top left header
\chead{}
\rhead{Project Proposal} % Top right header
\lfoot{} % Bottom left footer
\cfoot{} % Bottom center footer
\rfoot{Page\ \thepage\ of\ \protect\pageref{LastPage}} % Bottom right footer
\renewcommand\headrulewidth{0.4pt} % Size of the header rule
\renewcommand\footrulewidth{0.4pt} % Size of the footer rule

\renewcommand*{\thefootnote}{\fnsymbol{footnote}}

\setlength{\parskip}{.2\baselineskip}
%\setlength\parindent{0pt} % Removes all indentation from paragraphs

\title{
\textbf{Project Proposal} \\ \textsc{\textit{Sensing Semantic Information from Mobile Social Networks}} \\
}

\author{
	\textbf{\textit{Ting Zhang and Wen Yi}} \\
	School of Industiral Engineering, School of Electrical and Computer Engineering\\
	\texttt{zhan1013@purdue.edu, yi35@purdue.edu}
}

\date{\today}

%----------------------------------------------------------------------------------------

\begin{document}

\maketitle

%\thispagestyle{empty}

\section{Introduction}
Human society consists of extensive communications and interactions between individuals, via the use of mobile sensors, such as mobile phones, tablets and GPS. The understanding of individual relations from these sensors, can greatly facilitate and promote the interactions between individuals. 
For example, listing phone contacts in semantic orders according to the time and location when a person wants to make a phone call, would save the person both time and memory load, to find a specific contact from a phone book with tons of contacts based on alphabetic order. 
Further more, by understanding the friendship network from the global view, we can group people in the whole network into small communities in which people have closer friendships. This may serve as a friend grouping suggestion function inside the cellphone contact managing software.
Therefore, in this project, we will focus on the inference of friend relationships with the detection of communities using data collected from mobile phones. Detailed explanation of dataset is stated in section \ref{Data and Evaluation plan}.

\section{Related work}
\subsection{Relationship Inference}
Relationship inference in social networks has been studied in various fields and domains. In this context, we refer to friendship inference between pairs of individuals. Representing social networks with topology structures provides insights to predict relationships  between individuals based on topology and probability distribution of the links in the topology. \cite{liben-nowell_link-prediction_2007} proposed different measurements to compute the ''similarity'' between two nodes (individuals) in the graph, including the distance between two nodes, number of shared neighbors, and ''meta-approaches'' that integrate different measurements. Beyond topological structures, individual attribute and context information have also been utilized to facilitate the construction of relations between individuals. In the study from \cite{taskar_link_2004}, correlations between individuals were constructed using user attributes with relational Markov Networks. For instance, they proposed a transitivity pattern that is useful in relationship prediction, where the presence of A-B relation and B-C relation promotes the probability of A-C relation. Context information, such as locations and periods of time, has also shown potentials to predict social ties. Using location information alone may not be a sufficient predictor. In \cite{crandall_inferring_2010}'s work, only 0.1\% of the relations were predicted with a confidence of 60\%. However, when network structure are analyzed together with location information, over 90\% friendship were detected with confidence over 80\%, illustrated in the study from \cite{sadilek_finding_2012}. Although location information alone is not a good indicator for friendship inference, a number of researches have indicated the importance between social ties and distance. The integration of location information and other features are also proved to be of high accuracy in friendship inference.

\subsection{Community Detection}
While friendship inferring are engaged to predict the  local relationships between individuals, community detection, from a global aspect of view, groups people into smaller subgroups with tighter relationships. 
In previous research such as the studies from \cite{mislove_you_2010} and \cite{xie_community_2011}, from people's friendship conditions in social network, community detection groups the people into overlap or distinct ''communities'', while the members of the same community known quite a few others from the same community or the members share the similar characteristics. Further more, studies from  \cite{mislove_you_2010} showing that, community detection can be used to infer the vacant profile information of people based on the profile information from other members in the same community. 

From the algorithm implementation view, the social network is usually defined as an undirected unweighted graph, while the vertexes represent the individuals, the edges represents the friendship between individuals. 
To change the social network into communities, various of problem statement and algorithm were posed. In the study of \cite{girvan_community_2002}, the problem statement was '' removing the edges which is likely to be the friendship across communities, until the left graph divided into unconnected components, while each component represent one community''. On the other hand, in the study of  \cite{newman_fast_2004}, the problem statement becomes '' start with each individual as a community, then merge the community into another one with greatest increase on the global modularity''.

Besides the classic community detection model, as our social network may contains friendship duplicate between individuals such as multiple phone calls between pair of individuals, we may also try to apply clustering algorithms on weighted graph representations.


\section{Problem formulation}
\subsection{Friendship Inference}
In the dataset we will use, a survey was conducted for all participants, asking them rate the friendship with all other participants from 1 to 5. These information will be used as ground truth for the friendship inference problem. In the standard setting, we will binarize the friendship ratings and formulate the output of this problem as a binary label (either friend or not friend). As an advanced setting, we will try to predict the level of friendship, therefore, the output label would be from 1 to 5. The input of this problem would be a vector of features extracted from dataset. Different features and combinations of features will be experimented to investigate the key factors that would represent friendship. In the dataset, there is information including meeting time, duration and location between two individuals. 
\subsection{Community Detection}
For community detection problem, the ideal community result we want to get is the group belongings of the individual. As provided by the dataset we use, there were 94 volunteer participate the entire data collecting process. 68 of them were colleagues working in the same building on campus (90\% graduate students, 10\% staff) while the remaining 26 participants were incoming students at the university’s business school. The above identification information for each individual is stored inside the dataset.

For the problem statement, we define the network between the participants as two distinct social network, phone call network and Bluetooth network. Each phone call or Bluetooth connection between individuals would be counted as an ''edge'' in the network. As both of the networks may contain duplicate friendships, we may firstly turns the relation condition between people as a binary variable, which is '' whether the two individuals have connection''. For the fundamental implementation, we would use the clustering and modularity algorithm. Then, for an advanced trial, we would combine the duplicate friendship as weighted edge, then apply K-means clustering on the entire network.
Finally, we will try to combine the phone call network together with the Bluetooth network, and see if there is any improvement.
\subsection{Relations between Frienship Inference and Community Detection}
Friendship inference and community detection are two directions in the research of exploring social networks. These two directions can be different but also helpful in each other. For example, the result of community detection could be served as a feature indicating the relation between two individuals. People in the same community may inherit more probabilities to be friends. In the other direction, the level of friendship between individuals can also worked as a feature to predict community labels. The relations between firenship inference and community detection will also be investigated in this project.

\section{Data and Evaluation Plan} \label{Data and Evaluation plan}
We will use a reality mining dataset from MIT media lab \cite{eagle_inferring_2009}. The dataset consists of phone logs of 94 subjects from September 2004 to June 2005. Among these 94 subjects, 68 were colleagues working in the same building (90\% were graduate students, while 10\% were staff). The remaining 26 subjects were incoming students from the business school. The dataset was collected from Nokia 6600 phones programmed to automatically run a log application as background process, including phone log, bluetooth and location. The format of each log is summarized as following:
\begin{itemize}
\item[*]
Phone log: (TIME)  20060720T211505  (DESCRIPTION)  Voice  call  (DIRECTION)  Outgoing  (DURATION 
seconds) 23 (NUMBER) 6175559821
\item[*]
Bluetooth: (TIME) 20060721T111222 devices: 000e6d2a3564 [Amy’s Phone]000e6d2b06ea [Jon’s PalmPilot] 
\item[*]
Location:(TIME) 20060721T111222 (CELL AREA) 24127, (CELL TOWER) 111, (SERVICE PROVIDER) AT\&T 
Wirel (USER DEFINED LOCATION NAME) My Office
\end{itemize}

To evalute the performance of friendship inference algorithm, we will check the four measurements: false negatives, true negatives, precision and recall. The baseline of the result would be only considering the result of comunity detection. If two individuals were in the same community, they would be considered as friends. 
For the evaluation function of the community detection algorithm, we will check the group label inside each community, then compute the purity of the community, which is the purity of the cluster shown as \(\dfrac{\#majority\ group\ member}{\#total\ group\ member}\). Then, we will make a weighted combination for the purity in each community according to the size of community, then use the final purity value as the evaluation of our community detection algorithm(A value between 0 and 1, the larger the better).
For the baseline of the result, we should make the whole experiment group as one cluster, then compute the purity value for comparison.

%\nocite{*}
\bibliographystyle{plainnat}
%\bibliographystyle{ieeetr}
\bibliography{CS578_project}

\end{document}
